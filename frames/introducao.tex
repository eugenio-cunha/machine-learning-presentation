\section{Introdução}

\subsection{Padrões}
    \begin{frame}[fragile]{Definição de Padrões}
        \begin{figure}[H]
        \begin{center}
            \includegraphics[scale=0.50]{images/padroes.png}
        \end{center}
        \end{figure}

        São perceptíveis \textbf{regularidades} que repetem-se de maneira
        \textbf{previsível} no  mundo ou em um artefato produzido pelo homem.
    \end{frame}

    \begin{frame}[fragile]{Os Padrões no Mundo}
        \begin{figure}[H]
        \begin{center}
            \includegraphics[scale=0.60]{images/padroes_mundo.png}
        \end{center}
        \end{figure}

        Mendes (2007) explica, em seu estudo sobre a matemática na
        \textbf{natureza}, a ocorrência da \textbf{sequência} de
        \textbf{Fibonacci} na Natureza é tão frequente que é difícil acreditar
        que é acidental \cite{mendes2007matematica}.
    \end{frame}

    \begin{frame}[fragile]{Os Padrões nos Artefatos}
        \begin{figure}[H]
        \begin{center}
            \includegraphics[scale=0.60]{images/arte_rupestre.png}
        \end{center}
        \end{figure}

        Ribeiro, L. (2007) em seu trabalho, classifica os estilos de pinturas
        rupestres do norte mineiro e sudoeste baiano 
        \cite{ribeiro2007repensando}.
    \end{frame}

\subsection{Aprendizado de Máquina}
    \begin{frame}[fragile]{Aprendizado de Máquina}
        \begin{figure}[H]
        \begin{center}
            \includegraphics[scale=0.50]{images/previsao.png}
        \end{center}
        \end{figure}

        Souto, Lorena, Delbem e Carvalho explicam, o Aprendizado de Máquina, 
        provê técnicas capazes de aprender automaticamente a partir dos dados 
        disponíveis e produzir hipóteses uteis \cite{de2003tecnicas}.

        % Monard e Baranauskas (2003) explica, Aprendizado de Máquina é uma 
        % subárea da inteligência artificial, o algoritmo toma decisões baseado 
        % em experiências acumuladas através da solução bem sucedida de problemas 
        % anteriores, ou seja, o algoritmo reconhece padrões em conjuntos de 
        % dados, generaliza e torna-se capaz de fazer previsões. 
  \end{frame}
  
  \begin{frame}[fragile]{Aprendizado de Máquina}
    \begin{table}[]
    \centering
    % \caption{My caption}
    \label{my-label}
    \begin{tabular}{l|c|c|c|c|c|c|}
    \cline{2-7}
                            & \textbf{Cor}           & \textbf{HEX} & \textbf{R}   & \textbf{G}   & \textbf{B}   & \textbf{Classe} \\ \hline
    \multicolumn{1}{|l|}{1} & \cellcolor[HTML]{FF0000} & \#FF0000     & \textit{255} & \textit{0}   & \textit{0}   & Quente          \\ \hline
    \multicolumn{1}{|l|}{2} & \cellcolor[HTML]{0000FF} & \#0000FF     & \textit{0}   & \textit{0}   & \textit{255} & Quente          \\ \hline
    \multicolumn{1}{|l|}{3} & \cellcolor[HTML]{00FF00} & \#00FF00     & \textit{0}   & \textit{255} & \textit{0}   & Quente          \\ \hline
    \multicolumn{1}{|l|}{4} & \cellcolor[HTML]{FAEBD7} & \#FAEBD7     & \textit{250} & \textit{235} & \textit{215} & Fria            \\ \hline
    \multicolumn{1}{|l|}{5} & \cellcolor[HTML]{EEEEE0} & \#EEEEE0     & \textit{238} & \textit{238} & \textit{224} & Fria            \\ \hline
    \multicolumn{1}{|l|}{6} & \cellcolor[HTML]{E0EEEE} & \#E0EEEE     & \textit{224} & \textit{238} & \textit{238} & Fria            \\ \hline
    \multicolumn{1}{|l|}{7} & \cellcolor[HTML]{8B2252} & \#8B2252     & \textit{139} & \textit{34}  & \textit{82}  & ?               \\ \hline
    \end{tabular}
    \end{table}
  \end{frame}

  \begin{frame}[fragile]{Aprendizado de Máquina}
    \begin{table}[]
    \centering
    % \caption{My caption}
    \label{my-label}
    \begin{tabular}{l|c|c|c|c|c|c|c|}
    \cline{2-8}
                            & \textbf{Cor}           & \textbf{HEX} & \textbf{R}   & \textbf{G}   & \textbf{B}   & \multicolumn{1}{l|}{\textbf{SOMA}} & \textbf{Classe} \\ \hline
    \multicolumn{1}{|l|}{1} & \cellcolor[HTML]{FF0000} & \#FF0000     & \textit{255} & \textit{0}   & \textit{0}   & \textit{255}                       & Quente          \\ \hline
    \multicolumn{1}{|l|}{2} & \cellcolor[HTML]{0000FF} & \#0000FF     & \textit{0}   & \textit{0}   & \textit{255} & \textit{255}                       & Quente          \\ \hline
    \multicolumn{1}{|l|}{3} & \cellcolor[HTML]{00FF00} & \#00FF00     & \textit{0}   & \textit{255} & \textit{0}   & \textit{255}                       & Quente          \\ \hline
    \multicolumn{1}{|l|}{4} & \cellcolor[HTML]{FAEBD7} & \#FAEBD7     & \textit{250} & \textit{235} & \textit{215} & \textit{700}                       & Fria            \\ \hline
    \multicolumn{1}{|l|}{5} & \cellcolor[HTML]{EEEEE0} & \#EEEEE0     & \textit{238} & \textit{238} & \textit{224} & \textit{700}                       & Fria            \\ \hline
    \multicolumn{1}{|l|}{6} & \cellcolor[HTML]{E0EEEE} & \#E0EEEE     & \textit{224} & \textit{238} & \textit{238} & \textit{700}                       & Fria            \\ \hline
    \multicolumn{1}{|l|}{7} & \cellcolor[HTML]{8B2252} & \#8B2252     & \textit{139} & \textit{34}  & \textit{82}  & \textit{255}                       & Quente          \\ \hline
    \end{tabular}
    \end{table}
  \end{frame}

\subsection{Problema de Pesquisa}
  \begin{frame}[fragile]{Problema de Pesquisa}
  \end{frame}